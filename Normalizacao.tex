\documentclass[brazil,xcolor=table]{beamer}
\usepackage{beamerthemesplit}
\usepackage{listings}
\usepackage[brazil]{babel}
%\usepackage[T1]{fontenc}
\usepackage[utf8]{inputenc}  
\usepackage{ulem}  

\usepackage{adjustbox}
\usepackage{multirow}

\usepackage{textcomp}
\usepackage{url}
\usepackage{courier}
\usepackage{multicol}
\usepackage{hyperref}
\usepackage[font=bf,skip=\baselineskip]{caption}


\setbeamertemplate{navigation symbols}{}

\usetheme{Madrid}


\title{Normalização }
\author{André Luis Schwerz \\ 	andreluis@utfpr.edu.br}
\institute{Universidade Tecnológica Federal do Paraná}
\date{Banco de Dados 1 \\ 2017/2}

\newcommand{\bi}{\begin{itemize}}
\newcommand{\ei}{\end{itemize}}
\newcommand{\ew}[1]{\textit{#1}}
\newcommand{\hl}[1]{\textbf{#1}}
\newcommand{\ul}[1]{\underline{#1}}
\newcommand{\ft}[1]{\frametitle{#1}}
\newcommand{\fst}[1]{\framesubtitle{#1}}
\newcommand{\cmd}[1]{{\ttfamily \textbackslash #1}}
\newcommand{\env}[1]{{\ttfamily #1}}
\newcommand{\riso}{$\stackrel{..}{\smile}$}
\captionsetup[lstlisting]{font={scriptsize}}

\lstset{ %
  backgroundcolor=\color{white}, 
  basicstyle=\scriptsize\ttfamily,
  breakatwhitespace=false,        
  breaklines=true,                 
  captionpos=false,                    
  commentstyle=\color{green},   
  escapeinside={\%*}{*)},        
  extendedchars=true,              
  frame=single,                  
  keywordstyle=\color{blue},       
  language=SQL,                
  numbers=left,                    
  numbersep=5pt,                   
  numberstyle=\tiny\color{blue},
  rulecolor=\color{black},        
  showspaces=false,               
  showstringspaces=false,          
  showtabs=false,                  
  stepnumber=0,                    
  stringstyle=\color{purple},   
  tabsize=2,                      
  title=\lstname,                  
  morekeywords={not,\},\{,preconditions,effects, REAL },            
  deletekeywords={time}            
}

\DeclareGraphicsExtensions{.jpg,.pdf,.mps,.png,.gif,.eps}

\AtBeginSection[]
{
\begin{frame}<beamer>{Agenda}
\tiny
\tableofcontents[currentsection,currentsubsection, 
    hideothersubsections, 
    sectionstyle=show/shaded,
]
\end{frame}
}

%\colorlet{mycolor}{orange!80!black}% change this color to suit your needs
%
%\AtBeginSection[]{
%  \setbeamercolor{section in toc shaded}{use=structure,fg=structure.fg}
%  \setbeamercolor{section in toc}{fg=mycolor}
%  \setbeamercolor{subsection in toc shaded}{fg=black}
%  \setbeamercolor{subsection in toc}{fg=mycolor}
%  \frame<beamer>{\begin{multicols}{2}
%  \frametitle{Outline}
%  \setcounter{tocdepth}{2}  
%  \tableofcontents[currentsection,subsections]
%\end{multicols} 
% }
%}
%
%\setbeamercolor{author in head/foot}{fg=white}
%\setbeamercolor{title in head/foot}{fg=mycolor}
%\setbeamercolor{section in head/foot}{fg=mycolor}
%\setbeamertemplate{section in head/foot shaded}{\color{white!70!black}\insertsectionhead}
%\setbeamercolor{subsection in head/foot}{fg=mycolor}
%\setbeamertemplate{subsection in head/foot shaded}{\color{white!70!black}\insertsubsectionhead}
%\setbeamercolor{frametitle}{fg=white}
%\setbeamercolor{framesubtitle}{fg=white}

\setbeamertemplate{footline}[frame number]

\captionsetup[table]{labelformat=empty}

\begin{document}

\frame{\titlepage}

\frame{
	\ft{Agenda}
\tableofcontents
}


\section{1a. Forma Normal (1FN)}
\frame{
    \ft{Formas Normais}
    \fst{1a. Forma Normal (1FN)}
    \bi 
        \item Uma tabela está na 1FN se, e somente se, todas as colunas tiverem apenas valores atômicos, ou seja, se cada coluna só puder ter um valor para cada linha na tabela.
        \bi 
            \item Não pode conter atributo composto
            \item Não pode conter atributo multi-valorado
            \item Não pode conter conjuntos de atributos repetidos descrevendo a mesma característica
        \ei     
    \ei 
}


\frame{
    \ft{Formas Normais}
    \fst{Exemplo}
%\cellcolor[HTML]{C0C0C0}8
    \bi 
        \item Exemplo de relação que NÃO está na 1FN
        \item Suponha a relação:
        \bi 
            \item PessoaCurso(Nome, Cidade, ID, (Curso)*)
            \item O asterisco (*) significa que cada Pessoa terá um grupo de Cursos
        \ei
    \ei 
   \begin{table}[]
\centering
\begin{tabular}{|l|l|l|m{4cm}|}
\hline
\multicolumn{1}{|c|}{{\bf Nome}} & \multicolumn{1}{c|}{{\bf Cidade}} & \multicolumn{1}{c|}{{\bf ID }} & \multicolumn{1}{c|}{{\bf Cursos}} \\ \hline
Artur   &   São Paulo   &   999  & Programador                       \\ \hline
Ana     &   Londrina    &   777  & Operador, programador             \\ \hline
Carlos  &   Araruna     &   888  & Analista, programador, operador   \\ \hline
Paulo   &   Maringá     &   555  & Operador, analista                \\ \hline
\end{tabular}
\end{table}
    \bi 
        \item O atributo Cursos contém valores não atômicos
    \ei 
}

\frame{
     \ft{Formas Normais}
    \fst{Mais exemplo}
%\cellcolor[HTML]{C0C0C0}8
    \bi 
        \item Exemplo de relação que NÃO está na 1FN
    \ei 
   \begin{table}[]
\centering
\begin{tabular}{|l|l|l|l|l|l|}
\hline
\multicolumn{1}{|c|}{{\bf Nome}} & \multicolumn{1}{c|}{{\bf Cidade}} & \multicolumn{1}{c|}{{\bf ID }} & \multicolumn{1}{c|}{{\bf Curso 1}} & \multicolumn{1}{c|}{{\bf Curso 2}} & \multicolumn{1}{c|}{{\bf Curso 3}}                 \\ \hline
Artur   &   São Paulo   &   999  & Programador   &               &          \\ \hline
Ana     &   Londrina    &   777  & Operador      & Programador   &            \\ \hline
Carlos  &   Araruna     &   888  & Analista      & Programador   & Operador  \\ \hline
Paulo   &   Maringá     &   555  & Operador      & Analista      &            \\ \hline
\end{tabular}
\end{table}
    \bi 
        \item Porque há  atributos repetidos do mesmo tipo: Curso 1, Curso 2 e Curso 3.
        \item As tuplas correspondentes à alunos com apenas um ou dois cursos terão valores nulos para alguns atributos.
        \item Como representar uma pessoa com mais do que três cursos?
    \ei 
}

\frame{
     \ft{Formas Normais}
    \fst{Como decompor?}
    \bi 
        \item Pessoa (Nome, Cidade, \ul{ID})
        \item PessoaCurso(\ul{ID}, \ul{Curso})    
        
    \ei
    
    \begin{minipage}{0.49\textwidth}


\begin{table}[]
\centering
\begin{tabular}{|l|l|l|}
\hline
\multicolumn{1}{|c|}{{\bf Nome}} & \multicolumn{1}{c|}{{\bf Cidade}} & \multicolumn{1}{c|}{{\bf ID }} \\ \hline
Artur   &   São Paulo   &   999 \\ \hline
Ana     &   Londrina    &   777 \\ \hline
Carlos  &   Araruna     &   888 \\ \hline
Paulo   &   Maringá     &   555 \\ \hline
\end{tabular}
\end{table}


\end{minipage}
\begin{minipage}{0.49\textwidth}

    \begin{table}[]
\centering
\begin{tabular}{|l|l|}
\hline
\multicolumn{1}{|c|}{{\bf ID}} & \multicolumn{1}{c|}{{\bf Curso}} \\ \hline
 999    & Programador   \\ \hline
 777    & Operador      \\ \hline
 777    & Programador   \\ \hline 
 888    & Analista      \\ \hline
 888    & Programador   \\ \hline
 888    & Operador      \\ \hline
 555    & Operador      \\ \hline
 555    & Analista      \\ \hline
\end{tabular}
\end{table}


\end{minipage}
}

\section{2a. Forma Normal (2FN)}
\frame{
     \ft{Formas Normais}
    \fst{2a. Forma Normal (2FN)}
    \bi 
        \item Uma tabela está na 2FN se, e somente se, ela estiver na 1FN e os atributos não-chaves forem totalmente dependentes da chave primária. Um atributo será totalmente dependente da chave primária se estiver no lado direito de um DF que tem no lado esquerdo a própria chave primária ou algo que possa ser derivado da chave primária usando a transitividade das DFs. 
        \bi
            \item Não pode conter dependência parcial.
        \ei
    \ei
}

\frame{
     \ft{Formas Normais}
    \fst{Exemplos}
    \bi 
        \item Exemplo de relação que NÃO está na 2FN
        \bi
            \item Analise a relação, Pedido(\ul{NomeFornecedor, CodPeça}, Cidade, Quantidade) em que:
            \bi
                \item \{NomeFornecedor, CodPeça\} é a chave primária
                \item \{NomeFornecedor\} $\rightarrow$ \{Cidade\}
            \ei
        \ei
    \ei

\begin{table}[]
\centering
\begin{tabular}{|l|c|l|c|}
\hline
\multicolumn{1}{|c|}{{\bf NomeFornecedor}} & \multicolumn{1}{c|}{{\bf CodPeça}} & \multicolumn{1}{c|}{{\bf Cidade }} & \multicolumn{1}{c|}{{\bf Quantidade }}\\ \hline
Empresa A   &  1    &   Maringá  & 100 \\ \hline
Empresa A   &  2    &   Maringá  & 200 \\ \hline
Empresa A   &  3    &   Maringá  & 300 \\ \hline
Empresa B   &  1    &   Londrina & 400 \\ \hline
Empresa B   &  3    &   Londrina & 500 \\ \hline
\end{tabular}
\end{table}
 
    
}

\frame{
     \ft{Formas Normais}
    \fst{Exemplos}
    \bi 
        \item Algumas anomalias
        \bi
            \item {\bf Inserção}: Não é possível inserir um fornecedor sem que ele forneça alguma peça (peça faz parte da chave).
            \item {\bf Eliminação}: Se, por exemplo, a empresa B deixar de fornecer as peças 1 e 3, a informação sobre a cidade desse fornecedor será perdida.
            \item {\bf Modificação}: Supondo que um fornecedor muda de cidade. Atualizar R significa atualizar todas as tuplas desse fornecedor.
        \ei
    \ei
\begin{table}[]
\centering
\begin{tabular}{|l|c|l|c|}
\hline
\multicolumn{1}{|c|}{{\bf NomeFornecedor}} & \multicolumn{1}{c|}{{\bf CodPeça}} & \multicolumn{1}{c|}{{\bf Cidade }} & \multicolumn{1}{c|}{{\bf Quantidade }}\\ \hline
Empresa A   &  1    &   Maringá  & 100 \\ \hline
Empresa A   &  2    &   Maringá  & 200 \\ \hline
Empresa A   &  3    &   Maringá  & 300 \\ \hline
Empresa B   &  1    &   Londrina & 400 \\ \hline
Empresa B   &  3    &   Londrina & 500 \\ \hline
\end{tabular}
\end{table}
}


\frame{
     \ft{Formas Normais}
    \fst{Como decompor?}
    \bi 
        \item Pedido(\ul{NomeFornecedor, CodPeça}, Quantidade)
        \item Fornecedor(\ul{NomeFornecedor}, Cidade)
    \ei
\begin{table}[]
\centering
\begin{tabular}{|c|c|c|}
\hline
\multicolumn{1}{|c|}{{\bf NomeFornecedor}} & \multicolumn{1}{c|}{{\bf CodPeça}} & \multicolumn{1}{c|}{{\bf Quantidade }}\\ \hline
Empresa A   &  1    & 100 \\ \hline
Empresa A   &  2    & 200 \\ \hline
Empresa A   &  3    & 300 \\ \hline
Empresa B   &  1    & 400 \\ \hline
Empresa B   &  3    & 500 \\ \hline
\end{tabular}
\end{table}

\begin{table}[]
\centering
\begin{tabular}{|c|l|}
\hline
\multicolumn{1}{|c|}{{\bf NomeFornecedor}} & \multicolumn{1}{c|}{{\bf Cidade}} \\ \hline
Empresa A   & Maringá \\ \hline
Empresa B   & Londrina \\ \hline
\end{tabular}
\end{table}
}

\frame{
    \ft{Formas Normais}
    \fst{2a. Forma Normal (2FN)}
    \bi 
        \item Resumindo, para vermos se uma relação está na 2FN
        \bi
            \item Identificamos a chave da tabela. Se a chave for apenas um atributo, ou for constituída por todos os atributos da relação, então podemos concluir que está na 2FN.
            \item Se a chave for composta (tiver mais do que um atributo) verificamos se há atributos que não são chave e que dependem apenas de parte da chave. Se não houver, então está na 2FN
        \ei
        \item Caso contrário, temos que decompor.
    \ei
}
\section{3a. Forma Normal (3FN)}
\frame{
    \ft{Formas Normais}
    \fst{3a. Forma Normal (3FN)}
    \bi 
        \item Uma tabela está na 3FN, se, e somente se, para cada DF não trivial X $\rightarrow$ A (em que X e A são atributos simples ou compostos), uma das duas condições precisam ser mantidas:
        \bi
            \item O atributo X é uma superchave.
            \item O atributo A é membro de uma chave candidata.
        \ei
        
        \item Simplificando:
        \bi
            \item \hl{Uma relação está na 3FN se está na 2FN e nenhum dos atributos não chave depende de outro também não chave (dependência transitiva)!}
        \ei
    \ei
}

\frame{
    \ft{Formas Normais}
    \fst{Exemplos}
    \bi 
         
        \item Exemplo de relação que NÃO está na 3FN
      
        \item Analise a relação, Nota(\ul{NumNota},CodCliente, NomeCliente, CidadeCliente) em que:
        \bi
            \item  \{NumNota\} é a chave primária
            \item \{CodCliente\} $\rightarrow$ {NomeCliente}
            \item \{CodCliente\} $\rightarrow$ \{CidadeCliente\}
            \item Logo:
             \bi
                \item \{CodCliente\} $\rightarrow$ \{NomeCliente, CidadeCliente\}
            \ei
            \item E transitivamente:
             \bi
                \item \{NumNota\} $\rightarrow$ \{NomeCliente, CidadeCliente\}
            \ei
        \ei
    \ei

}
\frame{
     \ft{Formas Normais}
    \fst{Como decompor?}
    \bi 
        \item Cliente(\ul{CodCliente}, NomeCliente CidadeCliente)
        \item Nota(\ul{NumNota}, CodCliente)
    \ei
\begin{table}[]
\centering
\begin{tabular}{|c|c|c|}
\hline
\multicolumn{1}{|c|}{{\bf CodCliente}} & \multicolumn{1}{c|}{{\bf NomeCliente}} & \multicolumn{1}{c|}{{\bf CidadeCliente }}\\ \hline
1 & Maria &  Curitiba  \\ \hline
2 & Ana   &  Londrina  \\ \hline
3 & João  &  Maringá   \\ \hline
\end{tabular}
\end{table}

\begin{table}[]
\centering
\begin{tabular}{|c|c|}
\hline
\multicolumn{1}{|c|}{{\bf NumNota}} & \multicolumn{1}{c|}{{\bf CodCliente}} \\ \hline
1  & 1 \\ \hline
2  & 2 \\ \hline
3  & 2 \\ \hline
4  & 3 \\ \hline
5  & 3 \\ \hline
\end{tabular}
\end{table}
}


\frame{
     \ft{Exercício}
    
    \bi 
        \item Indique em forma normal a relação de se encontra e normalize para a 3FN:
        
        \item Arq-Alunos(\ul{CodAl}, NomeAl, (CodCurso, Ingresso, (CodDisc, SemDisCursada, Nota)))
    \ei
    
\footnotesize
\begin{table}[]
\centering
\begin{tabular}{|c|c|c|c|c|c|c|}
\hline
\multicolumn{1}{|c|}{{\bf CodAl}} & \multicolumn{1}{c|}{{\bf NomeAl}} &
\multicolumn{1}{c|}{{\bf CodCurso}} & \multicolumn{1}{c|}{{\bf Ingresso}} &
\multicolumn{1}{c|}{{\bf CodDisc}} & \multicolumn{1}{c|}{{\bf SemDisCursada}} & \multicolumn{1}{c|}{{\bf Nota}} \\ \hline

\multirow{4}{*}{1} & \multirow{4}{*}{Maria} & \multirow{2}{*}{1} & \multirow{2}{*}{2010-2} & 1  & 2010-1 & 6.0 \\ \cline{5-7}
                   &                        &                    &                         & 2  & 2010-2 & 7.0 \\ \cline{3-7}
                   &                        & \multirow{2}{*}{2} & \multirow{2}{*}{2011-1} & 3  & 2011-1 & 8.0 \\ \cline{5-7}
                   &                        &                    &                         & 4  & 2011-1 & 6.0 \\ \hline
\end{tabular}
\end{table}
}


\frame{
     \ft{Exercício}
    
    \bi 
        \item Indique em forma normal a relação de se encontra e normalize para a 3FN:
        
        \item ProjEmp(\ul{CodProj, CodEmp}, Nome, Cat, Sal, DataIni, TempAI)
        \bi
            \item \{CodProj, CodEmp\} é a chave primária
            \item \{CodProj, CodEmp\} $\rightarrow$ \{ DataIni, TempAI \}
            \item \{CodEmp\} $\rightarrow$ \{ Nome, Cat, Sal \}
            \item \{Cat\} $\rightarrow$ \{ Sal \}
            
        \ei
    \ei
\footnotesize
\begin{table}[]
\centering
\begin{tabular}{|c|c|c|c|c|c|c|}
\hline
\multicolumn{1}{|c|}{{\bf CodProj}} & \multicolumn{1}{c|}{{\bf CodEmp}} &
\multicolumn{1}{c|}{{\bf Nome}} & \multicolumn{1}{c|}{{\bf Cat}} &
\multicolumn{1}{c|}{{\bf Sal}} & \multicolumn{1}{c|}{{\bf DataIni}} & \multicolumn{1}{c|}{{\bf TempAI}} \\ \hline
    LSC001   & 2146 & João   & A1 & 4  & 01/11/91 & 24 \\ \hline
    LSC001   & 3145 & Sílvio & A2 & 4  & 02/10/91 & 24 \\ \hline
    LSC001   & 6126 & José   & B1 & 9  & 03/10/92 & 18 \\ \hline
    LSC001   & 1214 & Carlos & A2 & 4  & 04/10/92 & 18 \\ \hline
    LSC001   & 8191 & Mário  & A1 & 4  & 01/11/92 & 12 \\ \hline
    PAG02    & 8191 & Mário  & A1 & 4  & 01/05/93 & 12 \\ \hline
    PAG02    & 4112 & João   & A2 & 4  & 04/01/91 & 24 \\ \hline
    PAG02    & 6126 & José   & B1 & 9  & 01/11/92 & 12 \\ \hline
\end{tabular}
\end{table}
}


\end{document}